%!TEX TS-program = /usr/texbin/pdflatex
%%%%%%%%%%%%%%%%%%%%%%%%%%%%%%%%%%%%%%%%%%%%%%%%%%%%%%%%%%%%%%%%%%%%%%%%
%%%%%%%%%%%%%%%%%%%%%% Simple LaTeX CV Template %%%%%%%%%%%%%%%%%%%%%%%%
%%%%%%%%%%%%%%%%%%%%%%%%%%%%%%%%%%%%%%%%%%%%%%%%%%%%%%%%%%%%%%%%%%%%%%%%

%%%%%%%%%%%%%%%%%%%%%%%%%%%%%%%%%%%%%%%%%%%%%%%%%%%%%%%%%%%%%%%%%%%%%%%%
%% NOTE: If you find that it says                                     %%
%%                                                                    %%
%%                           1 of ??                                  %%
%%                                                                    %%
%% at the bottom of your first page, this means that the AUX file     %%
%% was not available when you ran LaTeX on this source. Simply RERUN  %% 
%% LaTeX to get the ``??'' replaced with the number of the last page  %% 
%% of the document. The AUX file will be generated on the first run   %%
%% of LaTeX and used on the second run to fill in all of the          %%
%% references.                                                        %%
%%%%%%%%%%%%%%%%%%%%%%%%%%%%%%%%%%%%%%%%%%%%%%%%%%%%%%%%%%%%%%%%%%%%%%%%

%%%%%%%%%%%%%%%%%%%%%%%%%%%% Document Setup %%%%%%%%%%%%%%%%%%%%%%%%%%%%

% Don't like 10pt? Try 11pt or 12pt
\documentclass[10pt]{article}

% This is a helpful package that puts math inside length specifications
\usepackage{calc}

% Layout: Puts the section titles on left side of page
\reversemarginpar

%
%         PAPER SIZE, PAGE NUMBER, AND DOCUMENT LAYOUT NOTES:
%
% The next \usepackage line changes the layout for CV style section
% headings as marginal notes. It also sets up the paper size as either
% letter or A4. By default, letter was used. If A4 paper is desired,
% comment out the letterpaper lines and uncomment the a4paper lines.
%
% As you can see, the margin widths and section title widths can be
% easily adjusted.
%
% ALSO: Notice that the includefoot option can be commented OUT in order
% to put the PAGE NUMBER *IN* the bottom margin. This will make the
% effective text area larger.
%
% IF YOU WISH TO REMOVE THE ``of LASTPAGE'' next to each page number,
% see the note about the +LP and -LP lines below. Comment out the +LP
% and uncomment the -LP.
%
% IF YOU WISH TO REMOVE PAGE NUMBERS, be sure that the includefoot line
% is uncommented and ALSO uncomment the \pagestyle{empty} a few lines
% below.
%

%% Use these lines for letter-sized paper
\usepackage[paper=letterpaper,
            %includefoot, % Uncomment to put page number above margin
            marginparwidth=1in,     % Length of section titles
            marginparsep=.05in,       % Space between titles and text
            margin=1in,               % 1 inch margins
            includemp]{geometry}

%% Use these lines for A4-sized paper
%\usepackage[paper=a4paper,
%            %includefoot, % Uncomment to put page number above margin
%            marginparwidth=30.5mm,    % Length of section titles
%            marginparsep=1.5mm,       % Space between titles and text
%            margin=25mm,              % 25mm margins
%            includemp]{geometry}

%% More layout: Get rid of indenting throughout entire document
\setlength{\parindent}{0in}

%% This gives us fun enumeration environments. compactitem will be nice.
\usepackage{paralist}

%% Reference the last page in the page number
%
% NOTE: comment the +LP line and uncomment the -LP line to have page
%       numbers without the ``of ##'' last page reference)
%
% NOTE: uncomment the \pagestyle{empty} line to get rid of all page
%       numbers (make sure includefoot is commented out above)
%
\usepackage{fancyhdr,lastpage}
\pagestyle{fancy}
%\pagestyle{empty}      % Uncomment this to get rid of page numbers
\fancyhf{}\renewcommand{\headrulewidth}{0pt}
\fancyfootoffset{\marginparsep+\marginparwidth}
\newlength{\footpageshift}
\setlength{\footpageshift}
          {0.5\textwidth+0.5\marginparsep+0.5\marginparwidth-2in}
\lfoot{\hspace{\footpageshift}%
       \parbox{4in}{\, \hfill %
%                    \arabic{page} of \protect\pageref*{LastPage} % +LP
%                    \arabic{page}                               % -LP
                    \hfill \,}}

% Finally, give us PDF bookmarks
\usepackage{color,hyperref}
\definecolor{darkblue}{rgb}{0.0,0.0,0.3}
\hypersetup{colorlinks,breaklinks,
            linkcolor=darkblue,urlcolor=darkblue,
            anchorcolor=darkblue,citecolor=darkblue}

%%%%%%%%%%%%%%%%%%%%%%%% End Document Setup %%%%%%%%%%%%%%%%%%%%%%%%%%%%


%%%%%%%%%%%%%%%%%%%%%%%%%%% Helper Commands %%%%%%%%%%%%%%%%%%%%%%%%%%%%

% The title (name) with a horizontal rule under it
%
% Usage: \makeheading{name}
%
% Place at top of document. It should be the first thing.
\newcommand{\makeheading}[1]%
        {\hspace*{-\marginparsep minus \marginparwidth}%
         \begin{minipage}[t]{\textwidth+\marginparwidth+\marginparsep}%
                {\large \bfseries #1}\\[-0.15\baselineskip]%
                 \rule{\columnwidth}{1pt}%
         \end{minipage}}

% The section headings
%
% Usage: \section{section name}
%
% Follow this section IMMEDIATELY with the first line of the section
% text. Do not put whitespace in between. That is, do this:
%
%       \section{My Information}
%       Here is my information.
%
% and NOT this:
%
%       \section{My Information}
%
%       Here is my information.
%
% Otherwise the top of the section header will not line up with the top
% of the section. Of course, using a single comment character (%) on
% empty lines allows for the function of the first example with the
% readability of the second example.
\renewcommand{\section}[2]%
        {\pagebreak[2]\vspace{1.3\baselineskip}%
         \phantomsection\addcontentsline{toc}{section}{#1}%
         \hspace{0in}%
         \marginpar{
         \raggedright \scshape #1}#2}

% An itemize-style list with lots of space between items
\newenvironment{outerlist}[1][\enskip\textbullet]%
        {\begin{itemize}[#1]}{\end{itemize}%
         \vspace{-.6\baselineskip}}

% An environment IDENTICAL to outerlist that has better pre-list spacing
% when used as the first thing in a \section 
\newenvironment{lonelist}[1][\enskip\textbullet]%
        {\vspace{-\baselineskip}\begin{list}{#1}{%
        \setlength{\partopsep}{0pt}%
        \setlength{\topsep}{0pt}}}
        {\end{list}\vspace{-.6\baselineskip}}

% An itemize-style list with little space between items
\newenvironment{innerlist}[1][\enskip\textbullet]%
        {\begin{compactitem}[#1]}{\end{compactitem}}

% To add some paragraph space between lines.
% This also tells LaTeX to preferably break a page on one of these gaps
% if there is a needed pagebreak nearby.
\newcommand{\blankline}{\quad\pagebreak[2]}

%%%%%%%%%%%%%%%%%%%%%%%% End Helper Commands %%%%%%%%%%%%%%%%%%%%%%%%%%%

%%%%%%%%%%%%%%%%%%%%%%%%% Begin CV Document %%%%%%%%%%%%%%%%%%%%%%%%%%%%

\begin{document}
\makeheading{William J. Woodall}

\section{Contact Information}
%
% NOTE: Mind where the & separators and \\ breaks are in the following
%       table.
%
% ALSO: \rcollength is the width of the right column of the table 
%       (adjust it to your liking; default is 1.85in).
%
\newlength{\rcollength}\setlength{\rcollength}{1.85in}%
%
\begin{tabular}[t]{@{}p{\textwidth-\rcollength}p{\rcollength}}
%\href{http://www.eng.auburn.edu/comp/}%
%     {Department of Computer Science and Software Engineering} & \\
%\href{http://www.auburn.edu/}{Auburn University}
%                           & \textit{Voice:} (256) 345-9938 \\
449 N Donahue            & \textit{Voice:} (256) 345-9938 \\
Apartment B13           & \textit{E-mail:}
\href{mailto:wjwwood@gmail.com}{wjwwood@gmail.com}\\
Auburn, AL 36832    & \textit{WWW:}
\href{http://www.williamjwoodall.com/}{williamjwoodall.com}\\
\end{tabular}

\section{Research Interests}
%
Mobile Robotics, Perception, Navigation and Control, Software Engineering/Education

\section{Education}
%
\href{http://www.auburn.edu/}{\textbf{Auburn University}}, 
Auburn, AL
\begin{outerlist}
\item[] M.S.,
        Master of Software Engineering, Expected Graduation Date: August 2012
        \begin{innerlist}
        \item GPA: 3.83
        \end{innerlist}
\item[] B.S., 
%        \href{www.eng.auburn.edu/comp/}{Computer Science and Software Engineering}, Senior
        Computer Science and Software Engineering
        % \begin{innerlist}
        % \item GPA: 2.99/3.5 \emph{cumlative/major}
        % \end{innerlist}

\end{outerlist}

\section{Professional Experience}

\href{http://gavlab.auburn.edu/}{\textbf{GAVLab (GPS and Vehicle Dynamics Lab), Auburn University}}, 
Auburn, AL
\begin{outerlist}

\item[] \textit{Graduate Research Assistant}%
       \hfill \textbf{Summer 2010 - Present}
       
\end{outerlist}

\blankline

%
\href{http://virticle.com}{\textbf{Virticle Corp.}}, 
Auburn, AL
\begin{outerlist}

\item[] \textit{Web Design Engineer}%
       \hfill \textbf{Summer 2009 - Spring 2010}
% \begin{innerlist}
% % \item Received and process help desk tickets for Maintaining Websites
% \item Design, Implement, Maintain Websites
% \end{innerlist}

\end{outerlist}

\blankline

\href{http://neptunetg.com}{\textbf{Neptune Technology Group}}, 
Tallassee, AL
\begin{outerlist}

\item[] \textit{IT CO-OP}%
        \hfill \textbf{Summer 2007, Spring 2008, Fall 2008}
% \begin{innerlist}
% \item Received and processed all help desk calls and tickets
% \item Troubleshot/Repaired/Installed Desktops, Networks, Servers
% % \item Installed most new machines and some servers
% % \item Helped others in a data center on mission critical servers
% \end{innerlist}

\end{outerlist}

% \blankline

% \textbf{\href{http://www.waynefarmsllc.com/}{Wayne Farms, LLC} Decatur Plant}, 
% Decatur, AL
% \begin{outerlist}
% 
% \item[] \textit{AutoCAD and Drafting}%
%         \hfill \textbf{May 2005 to August 2005}
% \begin{innerlist}
% \item Converted existing hard drafts into electronic drafts and cataloged them.
% \item Updated drafts to reflect current implementations.
% \end{innerlist}
% 
% \end{outerlist}

\section{Relevant Experience/ Achievements}
% 
% \href{http://openroboticsplatform.org/}{Open Robotics Platform} (ORP) \hfill \textbf{Spring 2009 to present}
% \begin{innerlist}
% \item Founder and Developer, open-source project based on Senior Design project
% \end{innerlist}
% 
% \blankline

% Autonomous Lawnmower Team, Auburn University \hfill \textbf{Summer 2009 to present}
\href{http://ocm.auburn.edu/featured_story/lawnmower.html}{Autonomous Lawnmower Team}, Auburn University \hfill \textbf{Summer 2009 to present}
\begin{innerlist}
\item \href{http://eng.auburn.edu/news/2010/06/automow-2010-competition.html}{Third Place in 2010}; Second Place in 2011
\end{innerlist}

\blankline

\href{http://www.ros.org/wiki/fuerte/Planning/OSX}{ROS for OS X SIG Coordinator} \hfill \textbf{Fall 2011 to Present}

\blankline

% \href{http://proteus.ausparc.com}{Proteus Project (UAV's and Mobile Robotics)} \hfill \textbf{Spring 2010 to August 2010}
% 
% \blankline

Nasa Lunar Excavator Competition Team, Auburn University \hfill \textbf{Fall 2009 - Spring 2010}
\begin{innerlist}
\item \href{http://www.nasa.gov/offices/education/centers/kennedy/technology/lunabotics.html#Press}
           {Second Place Overall and First Place Systems Engineering Paper in 2010 Competition}
\end{innerlist}

\blankline

\href{http://sparc.eng.auburn.edu/}{Student Projects and Research Committee} (SPaRC) \hfill \textbf{Fall 2007 to Spring 2010}
\begin{innerlist}
\item \href{http://williamjwoodall.com/wp-content/uploads/2011/06/2009-volume-19-issue-1-ieee-only.pdf}
           {Project leader for 2009 hardware team; First Place at the IEEE SouthEastCON '09}
\end{innerlist}

\blankline

Vice-Chair \href{http://www.eng.auburn.edu/ieee/}{Auburn IEEE Student Branch} \hfill \textbf{May 2008 to May 2010}
% \begin{innerlist}
% \item Organized schedules and invited companies to meetings and events
% \end{innerlist}

\blankline

Auburn \href{http://www.bestinc.org/}{BEST Robotics} Student Alumni Organization \hfill \textbf{Fall 2005 to present}
% \begin{innerlist}
% \item Team mentor for \href{http://www.lee-scott.org/lsarobotics/Site/The_Warriors.html}{Lee-Scott Academy BEST Robotics Team} 
% \item Technical Expert on the floor (A-Team), at both regional and national competitions
% \end{innerlist}

\section{Technical Skills} 
%
% Experience with software and hardware related to robotics and information technology
% 
% \blankline

Code Portfolios: \href{http://github.com/wjwwood}{http://github.com/wjwwood}, \href{https://github.com/Auburn-Automow/}{https://github.com/Auburn-Automow/}

\blankline

\textit{Programming}: Python, C++, Java, Ruby, C, javascript, PHP, ActionScript, SQL \\
\null \hfill \textit{note: Languages listed in order of proficiency}

\textit{Frameworks/Libraries}: CMake, ROS, PCL, OpenCV, Qt, wxWidgets, VTK, Boost, MOOS \\

% \blankline

\textit{Applications}: \LaTeX{}, Textmate, vim, GIT, Mercurial, SVN, bash, Trac, Redmine, Basecamp, SolidEdge, AutoCAD, Solidworks, Blender, Photoshop \\

% \blankline

\textit{Operating Systems}: Windows IT, Linux Embedded, Mac OS X Personally
% \begin{innerlist}
    % \item Windows IT Experience, Linux Embedded Experience, Mac OS X Personal Computer
% \item Advanced Microsoft Windows XP/Vista/Server deployment knowledge.
% \item Advanced Linux/BSD knowledge in customized embedded solutions (Gentoo/OpenBSD)
% \item Apple OS X user for three years, OS of choice.
% \end{innerlist}

\end{document}

%%%%%%%%%%%%%%%%%%%%%%%%%% End CV Document %%%%%%%%%%%%%%%%%%%%%%%%%%%%%
